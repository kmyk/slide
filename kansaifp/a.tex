\documentclass[pdflatex,12pt]{beamer}
\usetheme{CambridgeUS}
\usepackage[whole]{bxcjkjatype}
\usepackage[all]{xy}
\usepackage{listings}

\title{CPLしませんか}
% \author{大中 公幸}
\date[2015 7/4]{関西関数型道場 2015 7/4}

\begin{document}

\frame{\maketitle}

% \begin{frame}{誰}
%     \begin{flushright}
%         \includegraphics[scale=0.4]{icon.png}
%     \end{flushright}
%     \vspace{-80pt}
%     \Large
%     \begin{enumerate}
%         \item 大中 公幸
%         \item @solo\_rab
%         \item 神戸大学 B2
%         \item プログラミングそのものに興味がある
%         \item 競技プログラミングもする
%         \item 数学もする
%     \end{enumerate}
% \end{frame}

\begin{frame}{言いたいこと}
    \centering
    \scalebox{3.0}{CPLやろう}
\end{frame}

\begin{frame}{述語論理、三角関数、行列}
    \Large
    \begin{enumerate}
        \item 定理証明系 $\to$ 述語論理
        \item ゲーム製作 $\to$ 三角関数、行列
        \item 競技プログラミング $\to$ グラフ理論
    \end{enumerate}
\end{frame}

\begin{frame}{述語論理、三角関数、行列}
    \begin{enumerate}
        \item 定理証明系 $\to$ 述語論理
        \item ゲーム製作 $\to$ 三角関数、行列
        \item 競技プログラミング $\to$ グラフ理論
        \item % required
    \centering
    \scalebox{4.0}{?? $\to$ 圏論}
    \end{enumerate}
\end{frame}

\begin{frame}
    \centering
    \scalebox{3.0}{Haskell ?}
\end{frame}

\begin{frame}
    \centering
    \scalebox{8.0}{CPL}
\end{frame}

\begin{frame}
    \centering
    \Huge
    プログラミング言語CPLを \\ 圏論への踏み台にしよう
\end{frame}

\begin{frame}{CPLとは}
    \Large
    \begin{enumerate}
        \item \textcolor{red}Categorical \textcolor{red}Programming \textcolor{red}Language
        \item 圏論
        \item 必ず停止する
        \item とてもシンプル
        \item 日本語の情報が多い
        \item インストールが簡単 \\ \centering{(\texttt{\$ cabal install CPL})}
    \end{enumerate}
\end{frame}

\begin{frame}
    \centering
    \scalebox{3.0}{やるしかない!}
\end{frame}

\begin{frame}
    \centering
    \scalebox{3.0}{コード例}
\end{frame}

\begin{frame}[fragile]{直積型}
    \scalebox{1.4}{
        直積型の定義
    }
    \centering
    \begin{lstlisting}[basicstyle=\ttfamily]
    right object prod(A,B) with pair is
        pi1: prod -> A
        pi2: prod -> B
    end object;
    \end{lstlisting}
\end{frame}

\begin{frame}[fragile]{積}
    \scalebox{2.0}{
        積
    }
    \scalebox{2.0}{
        $$
        \xymatrix{
            & \ar[ld]_-f X \ar@{.>}[d]^-{\left<f,g\right>} \ar[rd]^-g & \\
            A & \ar[l]^-{\pi_1} A \times B \ar[r]_-{\pi_2} & B
        }
        $$
    }
\end{frame}

\begin{frame}[fragile]{積}
    \centering
    \scalebox{1.5}{
        $$
        \xymatrix{
            & \ar[ld]_-f X \ar@{.>}[d]|-{pair\left(f,g\right)} \ar[rd]^-g & \\
            A & \ar[l]^-{pi1} prod\left(A,B\right) \ar[r]_-{pi2} & B
        }
        $$
    }
    \normalsize
    \begin{lstlisting}[basicstyle=\ttfamily]
    right object prod(A,B) with pair is
        pi1: prod -> A
        pi2: prod -> B
    end object;
    \end{lstlisting}
\end{frame}

\begin{frame}[fragile]{関数swap}
    \centering
    \scalebox{1.5}{
        $$
        \xymatrix{
            & \ar[ld]_-{pi2} prod\left(B,A\right) \ar@{.>}[d]|-{swap} \ar[rd]^-{pi1} & \\
            A & \ar[l]^-{pi1} prod\left(A,B\right) \ar[r]_-{pi2} & B
        }
        $$
    }
    \normalsize
    \begin{lstlisting}[basicstyle=\ttfamily]
    let swap = pair(pi2,pi1)
    \end{lstlisting}
\end{frame}

\begin{frame}[fragile]{自然数}
    \centering
    \scalebox{1.8}{
        $$
        \xymatrix{
            1 \ar[r]^{zero} \ar[d] & \mathbb{N} \ar@{.>}[d]^{pr\left(o,s\right)} & \ar[l]_{succ} \mathbb{N} \ar[d] \\
            1 \ar[r]^{o} & X & \ar[l]_{s} X
        }
        $$
    }
    \normalsize
    \begin{lstlisting}[basicstyle=\ttfamily]
    left object nat with pr is
        zero: 1 -> nat
        succ: nat -> nat
    end object;
    \end{lstlisting}
\end{frame}

\begin{frame}[fragile]{自然数}
    \centering
    \scalebox{2.0}{
        $$
        \xymatrix{
            1 \ar@/_10pt/[rrr]_{2} \ar[r]^{zero} & \mathbb{N} \ar[r]^{succ} & \mathbb{N} \ar[r]^{succ} & \mathbb{N} \\
        }
        $$
    }
    \normalsize
    \begin{lstlisting}[basicstyle=\ttfamily]
    simp full succ.succ.zero;
    \end{lstlisting}
\end{frame}

\begin{frame}[fragile]{関数add}
    \centering
    \scalebox{1.7}{
        $$
        \xymatrix@C=36pt@R=36pt{
            1 \ar[r]^{zero} \ar[d] & \mathbb{N} \ar@{.>}[d]^{\widetilde{add}} & \ar[l]_{succ} \mathbb{N} \ar[d] \\
            1 \ar[r]^{\lambda \_. \lambda n. n} & \mathbb{N}^\mathbb{N} & \ar[l]_{\lambda f. succ \circ f} \mathbb{N}^\mathbb{N}
        }
        $$
    }
    \scalebox{1.5}{
        $$
        \xymatrix@C=36pt@R=36pt{
            \mathbb{N} \times \mathbb{N} \ar@{.>}[d]^{add} \\
            \mathbb{N} \\
        }
        $$
    }
    \begin{lstlisting}[basicstyle=\ttfamily]
    let add = ev.prod(pr(cur(pi2),
                         cur(succ.ev)),I)
    \end{lstlisting}
\end{frame}

\begin{frame}[fragile]{関数add}
    \centering
    \scalebox{2.0}{
        $$
        \xymatrix{
            1 \ar@/_10pt/[rr]_{5} \ar[r]^{\left<2,3\right>} & \mathbb{N} \times \mathbb{N} \ar[r]^{add} & \mathbb{N} \\
        }
        $$
    }
    \begin{lstlisting}[basicstyle=\ttfamily]
    simp full add.pair(succ.succ.zero,
                       succ.succ.succ.zero)
    \end{lstlisting}
\end{frame}

\begin{frame}
    \centering
    さらに、CPLと数学者向けの圏論の本の間として \newline
    \includegraphics[scale=0.2]{program-semantics.jpg}

    $\lambda$ 計算の意味論の話の流れの中で自然に圏論が導入されるし \newline
    CPLに関連する話も載ってるのでおすすめ
\end{frame}

\begin{frame}
    \centering
    \scalebox{2.0}{
        圏論プログラミング言語CPLを
    }
    \scalebox{2.0}{
        圏論への入口にしよう
    }
\end{frame}

\begin{frame}
    \centering
    \Large
    Thank You
\end{frame}

\end{document}
