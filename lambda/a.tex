\documentclass[pdflatex,17pt]{beamer}
\usetheme{Frankfurt}
%% use triangles for enumeration {{{
\setbeamertemplate{itemize items}[default]
\setbeamertemplate{enumerate items}[default]
% }}}
%% enable navigation circles {{{
\usepackage{remreset}
\makeatletter
\@removefromreset{subsection}{section}
\makeatother
\setcounter{subsection}{1}
% }}}
%% remove navigation symbols, which menu icons exists at bottom of slides {{{
\setbeamertemplate{navigation symbols}{}
% }}}
% enable japanese {{{
\usepackage[whole]{bxcjkjatype}
% }}}
%% draw diagrams {{{
\usepackage[all]{xy}
% }}}
% \usepackage{listings}

\title{$\lambda$計算入門}
\author{Kimiyuki Onaka}
\date{\today}

\begin{document}
\maketitle

\section{導入}
\begin{frame}{自己紹介}
    \begin{itemize}
        \item 名前: 大中 公幸
        \item 所属: 神戸大学~工学部~情報知能工学科~学部2年
        \item 得意分野: プログラミング, 計算機科学, 論理学
        \item twitter: @ki6o4
    \end{itemize}
\end{frame}

\begin{frame}{目標}
    \begin{itemize}
        \item 計算に興味を持ってもらう
        \item 簡単な計算ができるようになってもらう
        \begin{itemize}
            \item $1 + 1~=~??$
            \item $2 \times 3~=~??$
        \end{itemize}
    \end{itemize}
\end{frame}

\begin{frame}{構成}
    \begin{enumerate}
        \item $\lambda$記法の紹介
        \item $\lambda$式の定義
        \item $\beta$簡約の定義
        \item 自然数の定義
        \item 加算と乗算の定義
        \item 減算の定義
        \item combinator理論の紹介
    \end{enumerate}
\end{frame}

\begin{frame}{連絡}
    \begin{itemize}
        \item このslideは
    \end{itemize}
     {\rm http://github.com/kmyk/slide/lambda/a.pdf} \\ に置いてあります
    \begin{itemize}
        \item 定義の省略とか記号の濫用とかそこそこあります
    \end{itemize}
\end{frame}

\section{本論}

\subsection{$\lambda$記法}
\subsectionpage
\begin{frame}{$\lambda$記法}
    \begin{block}{通常の記法}
        $f(x) = x^2$
    \end{block}
    \begin{block}{mapsto}
        $f : x \mapsto x^2$
    \end{block}
    \begin{block}{$\lambda$記法}
        $f = \lambda x. x^2$
    \end{block}
\end{frame}

\begin{frame}{$\lambda$記法~利点}
    \begin{itemize}
        \item 名前を付けなくてよい
        \item 単独で使える、操作がしやすい
        \item e.g. $\lambda x. x^2$
    \end{itemize}
\end{frame}

\begin{frame}{$\lambda$記法~例1}
    \begin{enumerate}
        \item \begin{itemize}
            \item $f(x) := x^2$
            \item $f(4) = 16$
        \end{itemize}
        \item \begin{itemize}
            \item $f := \lambda x. x^2$
            \item $f(4) = 16$
        \end{itemize}
        \item \begin{itemize}
            \item $(\lambda x. x^2)(4) = 16$
        \end{itemize}
    \end{enumerate}
\end{frame}

\begin{frame}{$\lambda$記法~例2}
    \begin{enumerate}
        \item \begin{itemize}
            \item $f(x) := x^2$
            \item $g[h] := h(h(3))$
            \item $g[f] = (3^2)^2 = 81$
        \end{itemize}
        \item \begin{itemize}
            \item $f := \lambda x. x^2$
            \item $g := \lambda h. h(h(3))$
            \item $g(f) = (3^2)^2 = 81$
        \end{itemize}
        \item \begin{itemize}
            \item $(\lambda h. h(h(3)))(\lambda x. x^2) = 81$
        \end{itemize}
    \end{enumerate}
\end{frame}

\subsection{$\lambda$式}
\subsectionpage
\begin{frame}{$\lambda$式~定義}
    $\lambda$式とは以下のように定義される
    \begin{itemize}
        \item 変数記号$x_0, x_1, x_2, \dots$は$\lambda$式である
        \item $M,N$が$\lambda$式のとき $(MN)$は$\lambda$式である
        \item $x$が変数で$M$が$\lambda$式のとき $(\lambda x. M)$は$\lambda$式である
        \item 以上によって定義されるもののみが$\lambda$式である
    \end{itemize}
\end{frame}

\begin{frame}{$\lambda$式~例}
    \begin{itemize}
        \item $(\lambda x_0. ((x_1 x_0) x_0))$
        \item $(\lambda x_1. (x_1 (x_1 x_0)))$
        \item $((x_3 x_4)(\lambda x_{103}. x_4))$
        \item $(\lambda x_1. (\lambda x_2. (\lambda x_3. x_9)))$
        \item $(\lambda x_1. (\lambda x_1. (\lambda x_1. x_1)))$
    \end{itemize}
\end{frame}

\begin{frame}{$\lambda$式~省略記法}
    \begin{itemize}
        \item 変数は適当に使う
        \item 分かる範囲で括弧は省略
        \item $f g h$は$(f g) h$を表す
        \item $\lambda x y z. w$は$\lambda x. (\lambda y. (\lambda z. w))$を表す
    \end{itemize}
\end{frame}

\begin{frame}{$\lambda$式~省略記法~例}
    \begin{itemize}
        \item $\lambda x. p x x$
        \item $\lambda h. h (h 3)$
        \item $f g (\lambda x. y)$
        \item $\lambda x y z. w$
        \item $\lambda x. (\lambda x. (\lambda x. x))$
    \end{itemize}
\end{frame}

\newcommand{\betato}{\underset{\beta}{\to}}
\newcommand{\betasto}{\underset{\beta}{\twoheadrightarrow}}
\subsection{$\beta$簡約}
\subsectionpage
\begin{frame}{$\beta$簡約~定義}
    $\beta$変換とは、 \\ 以下で定義される二項関係$\betato \subseteq \Lambda \times \Lambda$
    \begin{itemize}
        \item $(\lambda x. M)N \betato M[x:=N]$
        \item $M \betato N$ならば$\lambda x. M \betato \lambda x. N$、$MP \betato NP$、$PM \betato PN$
    \end{itemize}
\end{frame}

\begin{frame}{$\beta$簡約~定義}
    $M[x:=N]$は以下のように定義される
    \begin{itemize}
        \item $M \equiv x$ならば$N$
        \item $M \equiv y (y \neq x)$ならば$y$
        \item $M \equiv PQ$ならば $(P[x:=N])(Q[x:=N])$
        \item $M \equiv \lambda x. P$ならば$\lambda x. P$
        \item $M \equiv \lambda y. P (y \neq x)$ならば$\lambda y. P[x:=N]$
    \end{itemize}
    つまり、 \\ $M$中の自由な$x$を$N$で置き換えている
\end{frame}

\begin{frame}{$\beta$簡約~例1}
    \begin{enumerate}
        \item $(\lambda x. (\lambda y (\lambda z. x y)))vw$
        \item $\betato (\lambda y (\lambda z. v y))w$
        \item $\betato (\lambda z. v w)$
    \end{enumerate}
\end{frame}

\begin{frame}{$\beta$簡約~例2}
    \begin{enumerate}
        \item $(\lambda x. x x)(\lambda y. y)$
        \item $\betato (\lambda y. y)(\lambda y. y)$
        \item $\betato (\lambda y. y)$
    \end{enumerate}
\end{frame}

\begin{frame}{$\beta$簡約~例3}
    \begin{enumerate}
        \item $(\lambda x. x (\lambda x. x))y$
        \item $\betato y(\lambda x. x)$
    \end{enumerate}
\end{frame}

\begin{frame}{$\beta$簡約~例4}
    \begin{enumerate}
        \item $(\lambda x. x x)(\lambda x. x x)$
        \item $\betato (\lambda x. x x)(\lambda x. x x)$
        \item $\betato (\lambda x. x x)(\lambda x. x x)$
        \item $\dots$
    \end{enumerate}
\end{frame}

\begin{frame}{$\beta$簡約~例5 \\ \small Turingの不動点演算子$\Theta$}
    \begin{itemize}
        \item \small $X := \lambda x f. f (x x f)$
        \item \small $Y := XX$
    \end{itemize}
    \begin{enumerate}
        \item \small $YM$
        \pause
        \item \small $\equiv (\lambda x f. f (x x f))XM$
        \pause
        \item \small $\betato (\lambda f. f (XXf))M$
        \pause
        \item \small $\betato M(XXM)$
        \pause
        \item \small $\equiv M(YM)$
    \end{enumerate}
\end{frame}

\begin{frame}{$\beta$簡約~例5 \\ \small Turingの不動点演算子$\Theta$}
    \begin{itemize}
        \item \small $X := \lambda x f. f (x x f)$
        \item \small $Y := XX$
    \end{itemize}
    \begin{enumerate}
        \item \small $YM$
        \item \small $\betato M(YM)$
        \item \small $\betato M(M(YM))$
        \item \small $\betato M(M(M(YM)))$
        \item \small $\dots$
    \end{enumerate}
\end{frame}

\newcommand{\church}[1]{\lceil #1 \rceil}
\subsection{自然数}
\subsectionpage
\begin{frame}{自然数~定義 \\ \small Church数}
    自然数$n$に対し、Church数$\church{n}$を以下のように定義する
    \begin{itemize}
        \item $\church{n} := \lambda f x. \overbrace{f (f (f (\dots (f}^n x) \dots)))$
    \end{itemize}
\end{frame}

\begin{frame}{自然数~例}
    \begin{itemize}
        \item $\church{0} := \lambda f x. x$
        \item $\church{1} := \lambda f x. f x$
        \item $\church{2} := \lambda f x. f (f x)$
        \item $\church{3} := \lambda f x. f (f (f x))$
        \item $\dots$
    \end{itemize}
\end{frame}

\subsection{演算~加算と乗算}
\subsectionpage
\begin{frame}{演算~加算}
    このような準備の下で、足し算とは
    \begin{itemize}
        \item $P\church{m}\church{n} \betasto \church{m+n}$
    \end{itemize}
    を満たすような$\lambda$式$P$のこと
    \pause
    \vspace{1em}
    \begin{itemize}
        \item $+~:= \lambda m n f x. m f (n f x)$
    \end{itemize}
    はこれを満たす
\end{frame}

\begin{frame}{演算~加算~計算~$2+3$}
    \begin{enumerate}
        \item \small $+\church{2}\church{3}$
        \pause
        \item \small $\equiv (\lambda m n f x. m f (n f x))\church{2}\church{3}$
        \pause
        \item \small $\betasto \lambda f x. \church{2} f (\church{3} f x)$
        \pause
        \item \small $\equiv \lambda f x. \church{2} f ((\lambda f x. f (f (f x))) f x)$
        \pause
        \item \small $\betasto \lambda f x. \church{2} f (f (f (f x)))$
        \pause
        \item \small $\equiv \lambda f x. (\lambda f x. f (f x)) f (f (f (f x)))$
        \pause
        \item \small $\betato \lambda f x. (\lambda x. f (f x)) (f (f (f x)))$
        \pause
        \item \small $\betato \lambda f x. f (f (f (f (f x))))$
        \pause
        \item \small $\equiv \church{5}$
    \end{enumerate}
\end{frame}

\begin{frame}{演算~乗算}
    掛け算とは
    \begin{itemize}
        \item $P\church{m}\church{n} \betasto \church{m\times n}$
    \end{itemize}
    を満たすような$\lambda$式$P$のこと
    \pause
    \vspace{1em}
    \begin{itemize}
        \item $\times~:= \lambda m n f x. m (n f) x$
    \end{itemize}
    はこれを満たす
\end{frame}

\subsection{演算~減算}
\subsectionpage
\begin{frame}{演算~後者関数}
    後者関数とは
    \begin{itemize}
        \item $P\church{n} \betasto \church{n+1}$
    \end{itemize}
    を満たすような$\lambda$式$P$のこと
    \pause
    \vspace{1em}
    \begin{itemize}
        \item ${\rm succ}~:= \lambda n f x. f (n f x)$
    \end{itemize}
    はこれを満たす
\end{frame}

\begin{frame}{構造~対}
    対とは、ある$\lambda$式$x$、$y$があって
    \begin{itemize}
        \item $x(PMN) \betasto M$
        \item $y(PMN) \betasto N$
    \end{itemize}
    を満たすような$\lambda$式$P$のこと
    \pause
    \begin{itemize}
        \item $\pi_1 := \lambda x y. x$
        \item $\pi_2 := \lambda x y. y$
        \item ${\rm pair}~:= \lambda x y p. p x y$
    \end{itemize}
    はこれを満たす
\end{frame}

\begin{frame}{演算~前者関数}
    このような準備の下で、前者関数とは
    \begin{itemize}
        \item $P\church{0} \betasto \church{0}$
        \item $P\church{n+1} \betasto \church{n}$
    \end{itemize}
    を満たすような$\lambda$式$P$のこと
    \pause
    $$ \begin{array}{rl}
    {\rm pred}~:= \lambda n. \pi_2 (n & (\lambda p. {\rm pair} ({\rm succ} (\pi_1 p)) (\pi_1 p)) \\
                                      & ({\rm pair} \church{0} \church{0}))
    \end{array} $$
    はこれを満たす
\end{frame}

\begin{frame}{演算~前者関数~例}
    \scriptsize
    $$ \begin{array}{rl}
             & {\rm pred}\church{2} \\
    \betasto & \pi_2 (\church{2} (\lambda p. {\rm pair} ({\rm succ} (\pi_1 p)) (\pi_1 p)) \\
             & ~~~~              ({\rm pair} \church{0} \church{0})) \\
    \betasto & \pi_2 ((\lambda p. {\rm pair} ({\rm succ} (\pi_1 p)) (\pi_1 p)) \\
             & ~~~~   ((\lambda p. {\rm pair} ({\rm succ} (\pi_1 p)) (\pi_1 p)) \\
             & ~~~~              ({\rm pair} \church{0} \church{0})))) \\
    \betasto & \pi_2 ((\lambda p. {\rm pair} ({\rm succ} (\pi_1 p)) (\pi_1 p)) \\
             & ~~~~   (({\rm pair} ({\rm succ} (\pi_1 ({\rm pair} \church{0} \church{0}))) \\
             & ~~~~ ~~~~ (\pi_1 ({\rm pair} \church{0} \church{0}))))) \\
    \betasto & \pi_2 ((\lambda p. {\rm pair} ({\rm succ} (\pi_1 p)) (\pi_1 p)) \\
             & ~~~~   (({\rm pair} \church{1} \church{0}))) \\
    \betasto & \pi_2 ({\rm pair} \church{2} \church{1}) \\
    \betasto & \church{1} \\
    \end{array} $$
\end{frame}

\begin{frame}{演算~限定減算~定義}
    引き算とは
    \begin{itemize}
        \item $P\church{m}\church{n} \betasto \church{m\dot{-}n}$
    \end{itemize}
    を満たすような$\lambda$式$P$のこと
    \pause
    \vspace{1em}
    \begin{itemize}
        \item $\dot{-}~:= \lambda m n. n {\rm pred} m$
    \end{itemize}
    はこれを満たす
\end{frame}

\begin{frame}{演算~限定減算~例}
    \begin{enumerate}
        \item $\dot{-}\church{5}\church{3}$
        \pause
        \item $\betasto \church{3}{\rm pred}\church{5}$
        \pause
        \item $\betasto {\rm pred}({\rm pred}({\rm pred}\church{5}))$
        \pause
        \item $\dots$
        \pause
        \item $\dots$
        \pause
        \item $\dots$
        \pause
        \item $\betasto \church{2}$
    \end{enumerate}
\end{frame}

\subsection{combinator理論}
\begin{frame}{SKI combinators}
    \begin{itemize}
        \item $S := \lambda f g x. f x (g x)$
        \item $K := \lambda x y. x$
    \end{itemize}
\end{frame}

\begin{frame}{SKI combinators~例}
    \begin{itemize}
        \item $I := \lambda x. x$
        \item $I \equiv SKK$
    \end{itemize}
\end{frame}

\begin{frame}{SKI combinators~変換}
    \begin{itemize}
        \item $T[x] \to x$
        \item $T[MN] \to (T[M])(T[N])$
        \item $T[\lambda x. x] \to SKK$
        \item $T[\lambda x. y] \to Ky$
        \item $T[\lambda x. \lambda y. M(x)] \to T[\lambda x. T[\lambda y. M(x)]]$
        \item $T[\lambda x. M(x)N(x)] \to S(T[\lambda x. M(x))(T[\lambda x. N(x)])$
    \end{itemize}
\end{frame}

\subsection{参考文献}
\begin{frame}{参考文献}
    \scriptsize
    \begin{itemize}
        \item 高橋正子, 計算論, 近代科学社(1991)
        \begin{itemize}
        \scriptsize
            \item 計算可能性の話
            \item $\lambda$計算の体系と他の計算体系が等価であることとか
            \item 逆極限法を用いて、無限に計算し続けた結果の極限の話をする
        \end{itemize}
        \item 横内寛文, プログラム意味論, 共立出版(1994)
        \begin{itemize}
        \scriptsize
            \item $\lambda$式の意味論の話
            \item こちらも逆極限法
            \item cartesian閉圏による形式体系も載ってる
        \end{itemize}
        \item 中島玲二, 数理情報学入門, 朝倉書店(1982)
        \begin{itemize}
        \scriptsize
            \item 絶版
            \item 意味論
            \item 実際に用いられているプログラミング言語を見据えた感じ
            \item 逆極限法とかも載ってる
        \end{itemize}
    \end{itemize}
\end{frame}

\end{document}
